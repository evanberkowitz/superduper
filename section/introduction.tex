\section{Introduction}

\epigraph{The questions run too deep\\
For such a simple man\\
Won't you please, \\
Please tell me what we've learned?}{Supertramp, \textit{The Logical Song}}

Supermassive black holes have nonzero supermass, the quantity that couples to $\mathcal{N}=1$ supergravity, while in theories with more supercharges one may have superdupermassive black holes, and so on.
Just as the no-hair theorem tells us that non-rotating black holes can be described as massive and charged, the no-hair supertheorem tell us that non-rotating supermassive black holes can be described as supermassive and supercharged.
String theory provides hope to understand how to go beyond the semiclassical limit to describe in detail if and how black holes preserve unitarity, while superstring theory provides hope to understand the unitarity of supermassive black holes.

With the recent observation of normal black hole \cite{Abbott:2016blz,Abbott:2016nmj,Abbott:2017vtc,Abbott:2017oio,Abbott:2017gyy} and neutron star\cite{TheLIGOScientific:2017qsa} mergers via gravitational waves by the super-sensitive Advanced LIGO and VIRGO detectors, it is not unreasonable to expect gravitational-wave detections of supernovae of collapsing superstars or violent supermassive black hole astrophysical phenomena are on the horizon or, if you'll forgive the absurd pun, superhorizon.

Astrophysical observation seems to be the best opportunity to study this exotic physics.
Direct detection of superpartners of familiar particles, with nonzero supermass, was a potential discovery at the Superconducting Super Collider, superior to the LHC due to its higher center-of-mass and supercenter-of-supermass collision energy.
When the super-expensive Superconducting Super Collider was superseded in the Congressional budget by the International Superspace Superstation, physicists' superegos were superficially bruised and a superb opportunity to supersize our superintelligence was lost.\footnote{
Other supervenient inventions were no doubt lost as well.
With CERN as a guiding historical precedent, had the Superconducting Super Collider not been deemed superfluous we might have, by now, cruised our supercars down the Information Superduperhighway.}
Other direct-detection experiments such as Super-Kamiokande have successfully performed as real-time supernova monitors and detected associated neutrinos, but, unfortunately, no sneutrinos, which are expected in the Standard Supermodel.\footnote{ 
SuperK is a Cherenkov detector consisting of about 13,000 PMTs mounted inside a superstructure of 50 kilotons, or about half a Panamax supertanker, of ultrapure water.  The PMTs are arranged in $3\times4$ arrays termed `supermodules' \cite{Fukuda:2002uc}.
SuperK also takes advantage of `Super Control Headers', a `Super-Low Energy trigger', and a `Super Memory Partner Module' to aid its search.
}

Balloon experiments which monitor cosmic rays and hunt for their astrophysical origins such as SuperTIGER\cite{Binns:2014xpa} may also shed light on these phenomena.
Supertranslations of superpositions of supersymmetric supermassive black holes in minisuperspace might accrete supernovae at a super-Eddington rate; the violent phenomena should produce clear signatures.

Another superpotential avenue of discovery is super-low-energy spectroscopy.
Supersoldiering on beyond fine and hyperfine splittings, spectroscopic detections of superfine splittings may provide an opportunity for \st{ultra-} super-low energy precision experiments to determine the small shifts associated with otherwise virtually-unobservable virtual superpartners.

Henceforth, when we need to distinguish between mass and supermass we henceforth use the `super' superscript, $m^{\text{super}}$.  When referring to the supermass of a subsystem we use the `sub' subscript, $m^{\text{super}}_{\text{sub}}$.  This avoids a disastrous notation with `sub' superscripts and `super' subscripts.
