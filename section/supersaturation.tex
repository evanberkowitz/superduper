\section{Supersaturation}

\epigraph{Baby, baby, baby.
}{Carlos Santana, \textit{Since Supernatural}}

First-order phase transitions allow for systems to be diabatically manipulated into metastable states.
One may, for example, supercool a liquid past its freezing point or superheat it past its boiling point.
Once disturbed, such a system may experience a dramatic transition to its ground state.

A chemical solution may similarly become supersaturated.
Once the solution is saturated, additional solute would typically precipitate or fail to desolve.
However, under certain circumstances, one may supersaturate a solution, increasing the solute's concentration past its saturation point.
This can be accomplished by saturating a high-temperature solvent and subsequent cooling, for example.
Just as supersaturation is more solute in less spatial volume than expected, by Lorentz invariance dissolution can also happen in less time than expected, as observed in the endochronic, or superluminal, dissolution of resublimated thiotimoline\cite{asimov:1948,asimov:1953,asimov:1960,vernon:2022}.
This rare phenomenon can likely be ascribed to the action of superoperators\cite{Deutsch:1991nm}.

The human brain may experience saturation directly.
Semantic saturation, or semantic satiation, is when a word is repeated so often that it starts to sound peculiar or lose its semantic meaning.
We speculate that there may be words or phrases that allow \emph{semantic supersaturation}, where, despite being repeated so often, past the bounds of typical semantic saturation or good taste, meaning is retained.
We know no example and a recent experiment pursued by the authors failed with a promising candidate, documented in Reference~\cite{self}.
There have also been other natural experiments which aimed for semantic supersaturation but ultimately chickened out\cite{chicken} or buffaloed the reader into confusion\cite{buffalo}.
Nevertheless, we believe there is no fundamental obstacle to semantic supersaturation and any difficulties are ultimately superable.
It may be beyond the ability of most people, requiring a superhuman or, indeed, Superman\footnote{We do note, however, that Superman, being from the planet Krypton, is not human in the biological sense of the word.}\cite{tippett:2009}, but it seems unlikely that experiencing semantic supersaturation requires superpowers.
