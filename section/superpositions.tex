\section{Superpositions}
\epigraph{I am uninterested in gravity,\\
and superuninterested in supergravity.}{Sidney Coleman\cite{vanNieuwenhuizen:2016}}

Supersymmetry has long been understood apart from its role in supergravity.
Schr\"{o}dinger, for example, presented a solution\cite{10.2307/20490744} to the quantum-mechanical Coulomb problem using operator, or supersymmetric, techniques\cite{RevModPhys.23.21}.

In normal quantum mechanics, we know that position states may be expressed as superpositions of momentum states.
In supersymmetric quantum mechanics, superposition states may be expressed as supersuperpositions of supermomentum states, where the supermomentum indicates how quickly a particle moves through superspace.
Supermomentum and superposition are therefore conjugate variables, and the supermomentum operator generates supertranslations.

Bell proved normal quantum mechanics to be incompatible with local realism\cite{bell1964einstein}.
The superinduced theorem shows supersymmetric quantum mechanics to be incompatible with local superrealism.
Unfortunately, the theorem's supering does not close the superdeterminism loophole.

The race is on to build scalable quantum computers.
One strategy is to construct superconducting qubits from cuprates which, unfortunately, are not well-described by the simple Cooper pairing of BCS theory\cite{Bardeen:1957kj} or the super Cooper pairing of super Booper-Cooper-Scooper theory.
With experimental control, these systems may prove ripe for superdense coding\cite{PhysRevLett.69.2881} or pave the way towards quantum superemacy\cite{Preskill:2012tg}.
