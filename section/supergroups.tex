\section{Supergroups}

\epigraph{There is geometry in the humming of the strings,\\
there is music in the spacing of the spheres}{Pythagoras}


In 1966 supersymmetry was proposed to relate mesons to baryons\cite{doi:10.1143/PTP.36.1266} without much purchase but in the early 1970s was subsequently understood as a potential extension to spacetime symmetry with applications in quantum field theory and string theory \cite{Gervais:1971ji,Ramond:1971gb,Volkov:1973ix,Wess:1974tw}.
In this application, the supergroups were generated by Lie superalgebras.

Previously a supermajority of model builders held a near-superstitious preference for supersymmetric models to cure, for example, the hierarchy problem.
The lack of evidence of any low-energy supersymmetry has raised alarms, and theorists have scrambled to find new BSM theories compatible with experimental constraints.
One strategy is to invoke strong dynamics, using Yang-Mills or super-Yang-Mills.
The fermions in these theories are bound by the glue, or the superglue, into baryons, which are natural dark matter candidates.
These theories may allow entire dark nuclear sectors, with stable states of more than one dark baryon.
Depending on the theory's parameters, there may be an entire phenomenology of superallowed transitions and transitions prohibited by superselection rules.
Unfortunately, a nonperturbative description of these theories and a quantitative understanding of their properties requires lattice methods and advanced supercomputing.

In a remarkable example of convergent evolution, supergroups emerged independently in 1966.
The first known example was Cream, generated by Eric Clapton, Jack Bruce, and Ginger Baker\cite{supergroups}.
Further supergroups, such as Blind Faith, Crosby, Stills, Nasy \& Young, A\reflectbox{B}BA, and The Super Super Blues Band\cite{supersuperblues} were discovered and explored, largely through the late 1960s and early 1970s.
While often producing superlative music, many of these supergroups were transient phenomena.